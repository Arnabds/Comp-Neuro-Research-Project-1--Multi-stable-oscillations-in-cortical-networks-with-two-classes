\documentclass[11pt]{letter} 
\usepackage{hyperref}
\renewcommand{\familydefault}{\sfdefault}
\usepackage{graphics,graphicx}
\renewcommand{\familydefault}{\sfdefault}
\topmargin=-2cm 
\textheight=19.5cm 
\textwidth=15cm
\evensidemargin=0cm 
\oddsidemargin=0.5cm 

\let\raggedleft\raggedright 

\begin{document}

\begin{letter}{~} 


\signature{\vspace*{-0.5cm}Arnab Dey Sarkar\\
Bard Ermentrout}


\opening{Dear Editor,}
Please consider our paper ``Multi-stable oscillations in cortical networks with two classes of inhibition'' for publication in PLoS Computational Biology. In this paper, we extend the classic excitatory-inhibitory network for the generation of cortical rhythms to include both parvalbumin (PV) and somatostatin (SOM) classes of inhibitory neurons. We start with a spiking network and demonstrate that for a range of inputs into the excitatory cells, the network is capable of generating two distinct stable rhythms. We apply a recent exact mean-field reduction to the model network to obtain a 9-dimensional system that is amenable to bifurcation analysis. We use the analysis to find conditions for multi-rhythmicity. In addition, we show that for some sets of parameters, we are able to find regimes where the amplitude of a fast rhythm is slowly modulated.  Our work builds on two recent papers (Tahvili, et al, Cell Reports 2025 \& Edwards et al, Bioarxiv, 2024) in which the authors examine the effects of two classes of inhibition on cortical rhythms. Because we are able to reduce the large spiking model to a much simpler system, we are able to expose the full richness of this circuit. Because of the interest in the roles of different classes on inhibitory circuits in cortical networks, we think our paper is timely and will be of interest to others in the area of computational neuroscience. Thank you for your attention.



\closing{Sincerely,}

\end{letter}
\end{document}
